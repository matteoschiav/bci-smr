\section{Introduction}
Brain Computer Interface (BCI) is a technique exploiting brain activity to control external devices.
Brain activity is monitored by looking at the activation/deactivation of groups of neurons in particular areas of the brain.
This task can be fulfilled by measuring either the changes in its internal blood flow, using functional magnetic resonance imaging (fMRI), or its electromagnetic activity, using detectors sensitive to electrical (EEG) or magnetic (MEG) fields.
\subsection{Motor Imagery}
Motor imagery is a change in the sensorimotor rhythm (SMR) of the primary motor cortex that appears when the motor areas are activated (e.g. when a motor task is imagined).
This activation is typically characterized by a decrease in amplitude of the SMR signal.
This decrease is due to the random alignment of neuronal electric dipoles appearing when a certain area of the brain is activated.
By looking at the EEG signal in both the spatial and frequency domain, it is possible to discriminate the imagined movement of different parts of the body, information that can be used for BCI. \\
In this report, we will show an example of BCI using motor imagery.
